%% This defines the bibliography file (main.bib) and the bibliography style.
%% If you want to create a bibliography file by hand, change the contents of
%% this file to a `thebibliography' environment.  For more information 
%% see section 4.3 of the LaTeX manual.
\renewcommand{\bibname}{References}
\begin{thebibliography}{00}
\bibitem{SMTheory} M. K. Gaillard, P. D. Grannis, and F. J. Sciulli, ``The Standard Model of Particle Physics'', Rev. Mod. Phys. 71 (1999)
\bibitem{QCDRunning} C. D. Roberts, ``Nonperturbative effects in QCD at Finite Temperature and Density'', Phys. Part. Nucl. 30 (1999) 
\bibitem{AlphaTheoEx} P.A. Zyla et al. (Particle Data Group), ``Review of Particle Physics'', Prog. Theor. Exp. Phys. 2020, 083 C01 (2020)
\bibitem{QCDAsym} J. Gross and F. Wilczek, ``Ultraviolet behavior of non-abelian gauge theories'', Phys. Rev. Lett. 30, 1343 (1973)
\bibitem{LQCDProtonMass}  S. D�rr et al. ``Ab Initio Determination of Light Hadron Masses'', Science. 322 (5905): 1224 7 (2008)
\bibitem{ChiPT} N. Fettes, U.-G. Mei{\ss}ner, and S. Steininger, ``Pion-nucleon scattering in chiral perturbation theory I: Isospin-symmetric case'', Nucl. Phys. A 640 (1998) 
\bibitem{QCDFactorization} J. C. Collins,  D. E. Soper, and G. F. Sterman, ``Factorization of Hard Processes in QCD'', Adv. Ser. Direct. High Energy Phys. 5 (1989)
\bibitem{SHM} Francesco Becattini, ``What is the meaning of the statistical hadronization model?'', J. Phys. Conf. Ser. 5 (2005) 
\bibitem{LSM} B. Andersson, G. Gustafson, G. Ingelman, and T. Sj�strand,  ``Parton fragmentation and string dynamics'', Phys. Rep. 97 (1983)
\bibitem{QCM} R. J. Fries, V. Greco, and P. Sorensen ``Coalescence Models For Hadron Formation From Quark Gluon Plasma'', Ann. Rev. Nucl. Part. Sci. 58 (2008)
\bibitem{QCDExtreme} F. Wilczek, ``QCD In Extreme Conditions'', Contribution to: 9th CRM Summer School: Theoretical Physics at the End of the 20th Century, 567-636
\bibitem{QCDDiffConds} E. d'Enterria, David G., et al., ``CMS physics technical design report: Addendum on high density QCD with heavy ions'', J. Phys.G 34 (2007)
\bibitem{MLBThermal} E. Altman, ``Many-body localization and quantum thermalization'', Nat. Phys. 14, 979 - 983 (2018).
\bibitem{ADSCFTThermal} M. P. Heller, R. A. Janik, and P. Witaszczyk, ``'Characteristics of Thermalization of Boost-Invariant Plasma from Holography'', Phys. Rev. Lett. 108, 201602 (2012)
\bibitem{QCDThermal} G. Parisi, ``Some considerations on the Quark-Gluon Plasma'', Quark Matter 2018 Conference (2018)
\bibitem{QCDVacuum} 
\bibitem{QCDThemDyn} H.C. Chandola, G. Punetha, and H. Dehnen, ``Dual QCD thermodynamics and quark-gluon plasma'', Nucl. Phys. A 945 (2016) 
\bibitem{StockR} R. Stock, ``Relativistic Nucleus-Nucleus Collisions and the QCD Matter Phase Diagram'', In *Landolt-Boernstein I 21A: Elementary particles* 7
\bibitem{QCDVacMelt} T.D. Lee and G.C. Wick, ``Vacuum stability and vacuum excitation in a spin-0 field theory'', Phys. Rev. D9 2291(1974) 
\bibitem{ChiralTemperature} J.O. Andersen and T. Brauner, ``Linear sigma model at finite density in the 1/N expansion to next-to-leading order'', Phys .Rev. D 78:014030 (2008)
\bibitem{ChiralRestore} M. Asakawa amd K. Yazaki, ``Chiral Restoration at Finite Density and Temperature", Nucl. Phys. A 504 (1989) 
\bibitem{ChiralPaper} K. Fukushima, D.E. Kharzeev, and H.J. Warringa, ``The Chiral Magnetic Effect'', Phys. Rev. D 78 074033 (2008)
\bibitem{CMESignature} S. Shi, H. Zhang, D. Hou, and J. Liao, ``Signatures of Chiral Magnetic Effect in the Collisions of Isobars'', Phys. Rev. Lett. 125 (2020) 
\bibitem{RestoreCME} J. Zhao and F-Q. Wang, ``Experimental searches for the chiral magnetic effect in heavy-ion collisions'', Prog. Part. Nucl. Phys.107 (2019)
\bibitem{CMEFigPaper} D.E. Kharzeev, J. Liao, S. A. Voloshin, and G. Wang, ``Chiral Magnetic and Vortical Effects in High-Energy Nuclear Collisions --- A Status Report'', Prog. Part. Nucl. Phys. 88 (2016)
\bibitem{CMEExpResult} S. Choudhury, G. Wang, W. He, Y. Hu, and H.Z. Huang, ``Background evaluations for the chiral magnetic effect with normalized correlators using a multiphase transport model'', Eur. Phys. J. C 80 (2020)
\bibitem{Cornell} H. S. Chung, J. Lee, and D. Kang, ``Cornell potential parameters for S-wave heavy quarkonia'', J. Korean Phys. Soc. 52 (2018)
\bibitem{CornellEquation}
\bibitem{CSEff} J. Harris and B. Muller, ``The Search for the quark-gluon plasma'', Ann. Rev. Nucl. Part. Sci. 46 (1996) 
\bibitem{TDepCornell}  A. Dumitru, Y. Guo, A. M�csy, and M. Strickland, ``Quarkonium states in an anisotropic QCD plasma'', Phys. Rev. D 79 (2009) 
\bibitem{Hagedorn} R. Hagedorn, ``Statistical thermodynamics of strong interactions at high energies'', Nuovo Cim. , Suppl. 3 (1965)
\bibitem{HagedornDeconfine} J. Rafelski, ``Melting Hadrons, Boiling Quarks'', from Hagedorn Temperature to Ultra-Relativistic Heavy-Ion Collisions at CERN. Springer, Cham.
\bibitem{DeconfineTemp} C.A. Dominguez, ``Color Deconfinement in {QCD} at Finite Temperature'', Nucl. Phys. B Proc. Suppl.15 (1990)
\bibitem{CondensedQCD} K. Rajagopal and F. Wilczek, ``The Condensed matter physics of QCD'', part of At the frontier of particle physics. Handbook of QCD. Vol. 1-3 (2000)
\bibitem{SmallX} M.B. Gay Ducati, ``High Density QCD'', Braz. J. Phys. 31 (2001)
\bibitem{GluonWalls} D.E. Kharzeev, ``Hot and dense matter: from RHIC to LHC: Theoretical overview'', Nucl. Phys. A 827 (2009)
\bibitem{DenseColorField} L.D. McLerran, S. Schlichting, S. Sen, ``Space-Time Picture of Baryon Stopping in the Color-Glass Condensate'', Phys. Rev. D 99, 074009 (2019)
\bibitem{CGCPaper} F. Gelis, E. Iancu, and J. Jalilian-Marian, R. Venugopalan ``The Color Glass Condensate'', Ann. Rev. Nucl. Part. Sci. 60 (2010)
\bibitem{GSIntro} A. Deshpande, Z.-E. Meziani, and J.-W. Qiu, ``Towards the next QCD Frontier with the Electron Ion Collider'', EPJ W of Conf, 113, 05019 (2016) 
\bibitem{DGLAP1} V.N. Gribov and L.N. Lipatov, Sov. J. Nucl. Phys. 15 (1972) 438.
\bibitem{DGLAP2} G.Altarelli and G. Parisi, Nucl. Phys. B126 (1977) 298.
\bibitem{DGLAP3} Yu. L. Dokshitzer, Sov. Phys. JETP 46 (1977) 641.
\bibitem{BFKL} G.P. Salam, ``An Introduction to leading and next-to-leading BFKL'', Acta Phys.Polon.B 30 (1999)
\bibitem{JIMWLKBK} K. Rummukainen and H. Weigert, ``Universal features of JIMWLK and BK evolution at small x'', Nucl. Phys. A 739 (2004)
\bibitem{GluonSatuPlot} C. Marquet, ``Open questions in QCD at high parton density'', Nucl.Phys.A 904 - 905 (2013)
\bibitem{EICGSDIH} L. Zheng, E.C. Aschenauer, J.H. Lee, and B.-W. Xiao, ``Probing Gluon Saturation through Dihadron Correlations at an Electron-Ion Collider'', Phys. Rev. D 89, 074037 (2014)
\bibitem{IntroShadow} J Jalilian-Marian and X.N. Wang, ``Small x gluons in nuclei and hadrons'', Phys. Rev. D 60, 054016 (1999)
\bibitem{DenseQCD} V.P. Gon�alves ``QCD at high parton density'', Braz. J. Phys.  34 (2004)
\bibitem{ExShadow} F. Arleo and T. Gousset, ``Measuring gluon shadowing with prompt photons at RHIC and LHC'', Phys. Lett. B 660 (2008)
\bibitem{StatMechHadron} P. Huovinen and P. Petreczky , ``QCD Equation of State and Hadron Resonance Gas'', Nucl. Phys. A 837 (2010) 
\bibitem{EOSHadron} N. Sarkar and P. Ghosh , ``van der Waals hadron resonance gas and QCD phase diagram'', Phys. Rev. C 98, 014907 (2018) 
\bibitem{NuclearShadowing} Jamal. Jalilian-Marian and X.N. Wang, ``Shadowing of gluons in perturbative QCD: A comparison of different models'', Phys. Rev. D63, 096001 (2001)
\bibitem{StrongNuclear} E. Epelbaum, H.-W. Hammer, and U.G. Mei{\ss}ner, ``Modern theory of nuclear forces'', Rev. Mod. Phys. 81 (2009)
\bibitem{QGPCosmology} J. Rafelski, "Connecting QGP-Heavy Ion Physics to the Early Universe``, Nucl. Phys. B Proc. Suppl. 243-244 (2013)  
\bibitem{MITBag} S. M. Sanches Jr., F. S. Navarra, and D. A. Foga�a, ``The quark gluon plasma equation of state and the expansion of the early Universe'', Nucl. Phy. A 937 (2015)
\bibitem{QGPEOSRef} E.S. Fraga and A. Kurkela, ``Interacting quark matter equation of state for compact stars'', Astrophys. J. Lett. 781, L25 (2014)
\bibitem{ColorSuperconductor} M. G. Alford, K. Rajagopal, T. Schaefer, A. Schmitt ``Color superconductivity in dense quark matter'', Rev. Mod. Phys. 80 (2008)
%\bibitem{ColorSuperconductor} D.K. Hong, ``Aspects of color superconductivity'', Acta Phys. Polon. B 32 (2001)
\bibitem{CSCOccurrence} M. G. Alford, ``Color superconducting quark matter'', Ann. Rev. Nucl. Part. Sci. 51 (2001)

\bibitem{QCDFirstOrder} K. Rajagopal, ``Mapping the QCD phase diagram'', Nucl. Phys. A 661 (1999) 

\bibitem{EOSPhase} G. Odyniec on behalf of STAR Collaboration, ``Beam Energy Scan Program at RHIC (BES I and BES II) -- Probing QCD Phase Diagram with Heavy-Ion Collisions'', PoS CORFU2018 (2019) 


\bibitem{CriticalPointTH} Z. Fodor and S.D. Katz, ``Critical point of QCD at finite T and mu, lattice results for physical quark masses'', JHEP 04 050 (2004)

\bibitem{CriticalPointEX} S. Gupta, X. Luo, B. Mohanty, H. G. Ritter, N. Xu, ``Scale for the Phase Diagram of Quantum Chromodynamics'', Science 332 (2011)

\bibitem{SQMReview} R.X. Xu, ``Strange quark stars - A Review'', IAU Symp. 214 (2003)

\bibitem{SS1} Y.-Z. Fan, Y.-W. Yu, D. Xu, Z.-P. Jin, X.-F. Wu, D.-M. Wei, and B. Zhang, ``A supra-massive magnetar central engine for short GRB 130603B'', Astrophys. J. Lett. 779 (2013) 
\bibitem{SS2} Z. G. Dai, S. Q. Wang, J. S. Wang, L. J. Wang, and Y. W. Yu, ``The Most Luminous Supernova ASASSN-15lh: Signature of a Newborn Rapidly-Rotating Strange Quark Star'', Astrophys. J. 817 (2016)
\bibitem{SS3} 

\bibitem{RHICReport} D. Trbojevic and S. Peggs, ``Required Accuracy of the RHIC Circumference'', United States: N. p., Web. doi:10.2172/1119398 (1993)

\bibitem{AuStripping} M. J. Rhoades-Brown, ``The Heavy Ion Injection Scheme for RHIC'', Proc. of the Workshop on the RHIC Performance (1988)
\bibitem{FirstAuSource} D. B. Steski, J. Alessi, J. Benjamin, C. Carlson, M. Manni, P. Thieberger, and M. Wiplich, ``Operation of the Relativistic Heavy Ion Collider $Au^-$ ion source'', Review of Scientific Instruments 73, 797 (2002) 

\bibitem{RHICStrpDetail} D.B. Steski and P. Thieberger, ``Stripping foils at RHIC'', Nucl. Instrum. Meth. A 613 (2010) 
\bibitem{AuStripRef} P. Thieberger, L. Ahrens, J. Alessi, J. Benjamin, M. Blaskiewicz, J. M. Brennan, K. Brown, C. Carlson, C. Gardner, W. Fischer, D. Gassner, J. Glenn, W. Mac Kay, G. Marr, T. Roser, K. Smith, L. Snydstrup, D. Steski, D. Trbojevic, N. Tsoupas, V. Zajic, and K. Zeno, ``Improved gold ion stripping at 0.1 and 10 GeV/nucleon for the Relativistic Heavy Ion Collider'', Phys. Rev. ST Accel. Beams 11, 011001 (2008)
\bibitem{LHCReport} L. Evans, ``The Large Hadron Collider'', Phil. Trans. R. Soc. A 370 (2012) 
\bibitem{OORun} J. Brewer, A. Mazeliauskas, W. van der Schee, ``Opportunities of OO and pO collisions at the LHC'', CERN Theory Report: CERN-TH-2021-028 (2021) 
\bibitem{LHCStrip} M. Schaumann, R. Alemany-Fernandez, H. Bartosik, T. Bohl, R. Bruce, G-H Hemelsoet, S. Hirlaender, J. Jowett, V. Kain, M. Krasny, J. Molson, G. Papotti, M.S. Camillocci, H. Timko, and J. Wenninger, ``First partially stripped ions in the LHC (${}^{208}Pb^{81+}$)'' J. Phys. Conf. Ser. 1350, 012071 (2019)

\bibitem{CYWong} C.Y. Wong, ``Introduction to high-energy heavy ion collisions'', Singapore, Singapore: World Scientific (1994) 516 p

\bibitem{IPHICText} Z.-T. Liang and X.-N. Wang , ``Globally Polarized Quark-Gluon Plasma in Noncentral A + A Collisions'', Phys.Rev.Lett. 96, 039901 (2006)
\bibitem{CentDef} I. Altsybeev and V. Kovalenko, ``Classifiers for centrality determination in proton-nucleus and nucleus-nucleus collisions'', EPJ Web Conf. 137, 11001
\bibitem{ALICEZDC} P. Cortese, ``Performance of the ALICE Zero Degree Calorimeters and upgrade strategy'', J. Phys. Conf. Ser. 1162, 012006 (2019)
\bibitem{CMSZDC} Oliver Suranyi, ``Study of Very Forward Neutrons with the CMS Zero Degree Calorimeter'', Universe 5 10, 210 (2019)
\bibitem{ATLASZDC} P. Dmitrieva and I. Pshenichnov, ``On the performance of Zero Degree Calorimeters in detecting multinucleon events'', Nucl. Instrum. Meth. A 906 (2018)
\bibitem{CentPlot} M. L. Miller, K. Reygers, S. J. Sanders and P. Steinberg, ?Glauber modeling in high energy nuclear collisions,? Ann. Rev. Nucl. Part. Sci. 57, 205 (2007) 
\bibitem{Glauber} R. J. Glauber, ``Quantum Optics and Heavy Ion Physics'', Nucl. Phys. A 774 (2006)
\bibitem{Optical1} J. Chauvin, D. Bebrun, A. Lounis, and M. Buenerd, ``Low and intermediate energy nucleus-nucleus elastic scattering and the optical limit of Glauber theory'', Phys. Rev. C. 28, 1970 (1983)
\bibitem{Optical2} T. Wibig and D. Sobczynska, ``Proton-nucleus cross section at high energies'', J. Phys. G: Nucl. Part. Phys. 24, 2037 (1998)
\bibitem{NPartScaling} B. B. Back, ``Studies of multiplicity in relativistic heavy-ion collisions'', 	J.Phys.Conf.Ser. 5 (2000)
\bibitem{NCollScaling} A. Milov, ``Electroweak probes with ATLAS'', PoS High-pT2017 016 (2019)


\bibitem{QuarkoniaV} C. Quigg and J. L. Rosner, ``Quantum Mechanics with Applications to Quarkonium'', Phys. Rept. 56 167-235 (1979) 

\bibitem{STARJpsi} STAR Collaboration, ``Measurement of inclusive $J/\psi$ suppression in Au+Au collisions at $\sqrt{s_{NN}}$ = 200 GeV through the dimuon channel at STAR'', Phys. Lett. B 797, 134917 (2019)








































\end{thebibliography}  



\bibliographystyle{unsrturl}
\bibliography{refs}
