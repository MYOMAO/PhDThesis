%% This defines the bibliography file (main.bib) and the bibliography style.
%% If you want to create a bibliography file by hand, change the contents of
%% this file to a `thebibliography' environment.  For more information 
%% see section 4.3 of the LaTeX manual.
\renewcommand{\bibname}{References}
\begin{thebibliography}{00}
\bibitem{StandardModel} S. Weinberg, ``A Theory of Leptons'', Phys. Rev. Lett. 19 1264-1266 (1967)
\bibitem{SMTheory} M. K. Gaillard, P. D. Grannis, and F. J. Sciulli, ``The Standard Model of Particle Physics'', Rev. Mod. Phys. 71 (1999)
\bibitem{QCDRunning} C. D. Roberts, ``Nonperturbative effects in QCD at Finite Temperature and Density'', Phys. Part. Nucl. 30 (1999) 
\bibitem{AlphaTheoEx} P.A. Zyla et al. (Particle Data Group), ``Review of Particle Physics'', Prog. Theor. Exp. Phys. 2020, 083 C01 (2020)
\bibitem{QCDAsym} J. Gross and F. Wilczek, ``Ultraviolet behavior of non-abelian gauge theories'', Phys. Rev. Lett. 30, 1343 (1973)
\bibitem{LQCDProtonMass}  S. D�rr et al. ``Ab Initio Determination of Light Hadron Masses'', Science. 322 (5905): 1224 7 (2008)
\bibitem{ChiPT} N. Fettes, U.-G. Mei{\ss}ner, and S. Steininger, ``Pion-nucleon scattering in chiral perturbation theory I: Isospin-symmetric case'', Nucl. Phys. A 640 (1998) 
\bibitem{QCDFactorization} J. C. Collins,  D. E. Soper, and G. F. Sterman, ``Factorization of Hard Processes in QCD'', Adv. Ser. Direct. High Energy Phys. 5 (1989)
\bibitem{SHM} Francesco Becattini, ``What is the meaning of the statistical hadronization model?'', J. Phys. Conf. Ser. 5 (2005) 
\bibitem{LSM} B. Andersson, G. Gustafson, G. Ingelman, and T. Sj�strand,  ``Parton fragmentation and string dynamics'', Phys. Rep. 97 (1983)
\bibitem{QCM} R. J. Fries, V. Greco, and P. Sorensen ``Coalescence Models For Hadron Formation From Quark Gluon Plasma'', Ann. Rev. Nucl. Part. Sci. 58 (2008)
\bibitem{QCDExtreme} F. Wilczek, ``QCD In Extreme Conditions'', Contribution to: 9th CRM Summer School: Theoretical Physics at the End of the 20th Century, 567-636
\bibitem{QCDDiffConds} E. d'Enterria, David G., et al., ``CMS physics technical design report: Addendum on high density QCD with heavy ions'', J. Phys.G 34 (2007)
\bibitem{MLBThermal} E. Altman, ``Many-body localization and quantum thermalization'', Nat. Phys. 14, 979 - 983 (2018).
\bibitem{ADSCFTThermal} M. P. Heller, R. A. Janik, and P. Witaszczyk, ``'Characteristics of Thermalization of Boost-Invariant Plasma from Holography'', Phys. Rev. Lett. 108, 201602 (2012)
\bibitem{QCDThermal} G. Parisi, ``Some considerations on the Quark-Gluon Plasma'', Quark Matter 2018 Conference (2018)
\bibitem{QCDVacuum} 
\bibitem{QCDThemDyn} H.C. Chandola, G. Punetha, and H. Dehnen, ``Dual QCD thermodynamics and quark-gluon plasma'', Nucl. Phys. A 945 (2016) 
\bibitem{StockR} R. Stock, ``Relativistic Nucleus-Nucleus Collisions and the QCD Matter Phase Diagram'', In *Landolt-Boernstein I 21A: Elementary particles* 7
\bibitem{QCDVacMelt} T.D. Lee and G.C. Wick, ``Vacuum stability and vacuum excitation in a spin-0 field theory'', Phys. Rev. D9 2291(1974) 
\bibitem{ChiralTemperature} J.O. Andersen and T. Brauner, ``Linear sigma model at finite density in the 1/N expansion to next-to-leading order'', Phys .Rev. D 78:014030 (2008)
\bibitem{ChiralRestore} M. Asakawa amd K. Yazaki, ``Chiral Restoration at Finite Density and Temperature", Nucl. Phys. A 504 (1989) 
\bibitem{ChiralPaper} K. Fukushima, D.E. Kharzeev, and H.J. Warringa, ``The Chiral Magnetic Effect'', Phys. Rev. D 78 074033 (2008)
\bibitem{CMESignature} S. Shi, H. Zhang, D. Hou, and J. Liao, ``Signatures of Chiral Magnetic Effect in the Collisions of Isobars'', Phys. Rev. Lett. 125 (2020) 
\bibitem{RestoreCME} J. Zhao and F-Q. Wang, ``Experimental searches for the chiral magnetic effect in heavy-ion collisions'', Prog. Part. Nucl. Phys.107 (2019)
\bibitem{CMEFigPaper} D.E. Kharzeev, J. Liao, S. A. Voloshin, and G. Wang, ``Chiral Magnetic and Vortical Effects in High-Energy Nuclear Collisions --- A Status Report'', Prog. Part. Nucl. Phys. 88 (2016)
\bibitem{CMEExpResult} S. Choudhury, G. Wang, W. He, Y. Hu, and H.Z. Huang, ``Background evaluations for the chiral magnetic effect with normalized correlators using a multiphase transport model'', Eur. Phys. J. C 80 (2020)
\bibitem{Cornell} H. S. Chung, J. Lee, and D. Kang, ``Cornell potential parameters for S-wave heavy quarkonia'', J. Korean Phys. Soc. 52 (2018)
\bibitem{CornellEquation}
\bibitem{CSEff} J. Harris and B. Muller, ``The Search for the quark-gluon plasma'', Ann. Rev. Nucl. Part. Sci. 46 (1996) 
\bibitem{TDepCornell}  A. Dumitru, Y. Guo, A. M�csy, and M. Strickland, ``Quarkonium states in an anisotropic QCD plasma'', Phys. Rev. D 79 (2009) 
\bibitem{Hagedorn} R. Hagedorn, ``Statistical thermodynamics of strong interactions at high energies'', Nuovo Cim. , Suppl. 3 (1965)
\bibitem{HagedornDeconfine} J. Rafelski, ``Melting Hadrons, Boiling Quarks'', from Hagedorn Temperature to Ultra-Relativistic Heavy-Ion Collisions at CERN. Springer, Cham.
\bibitem{DeconfineTemp} C.A. Dominguez, ``Color Deconfinement in {QCD} at Finite Temperature'', Nucl. Phys. B Proc. Suppl.15 (1990)
\bibitem{CondensedQCD} K. Rajagopal and F. Wilczek, ``The Condensed matter physics of QCD'', part of At the frontier of particle physics. Handbook of QCD. Vol. 1-3 (2000)
\bibitem{SmallX} M.B. Gay Ducati, ``High Density QCD'', Braz. J. Phys. 31 (2001)
\bibitem{GluonWalls} D.E. Kharzeev, ``Hot and dense matter: from RHIC to LHC: Theoretical overview'', Nucl. Phys. A 827 (2009)
\bibitem{DenseColorField} L.D. McLerran, S. Schlichting, S. Sen, ``Space-Time Picture of Baryon Stopping in the Color-Glass Condensate'', Phys. Rev. D 99, 074009 (2019)
\bibitem{CGCPaper} F. Gelis, E. Iancu, and J. Jalilian-Marian, R. Venugopalan ``The Color Glass Condensate'', Ann. Rev. Nucl. Part. Sci. 60 (2010)
\bibitem{GSIntro} A. Deshpande, Z.-E. Meziani, and J.-W. Qiu, ``Towards the next QCD Frontier with the Electron Ion Collider'', EPJ W of Conf, 113, 05019 (2016) 
\bibitem{DGLAP1} V.N. Gribov and L.N. Lipatov, Sov. J. Nucl. Phys. 15 (1972) 438.
\bibitem{DGLAP2} G.Altarelli and G. Parisi, Nucl. Phys. B126 (1977) 298.
\bibitem{DGLAP3} Yu. L. Dokshitzer, Sov. Phys. JETP 46 (1977) 641.
\bibitem{BFKL} G.P. Salam, ``An Introduction to leading and next-to-leading BFKL'', Acta Phys.Polon.B 30 (1999)
\bibitem{JIMWLKBK} K. Rummukainen and H. Weigert, ``Universal features of JIMWLK and BK evolution at small x'', Nucl. Phys. A 739 (2004)
\bibitem{GluonSatuPlot} C. Marquet, ``Open questions in QCD at high parton density'', Nucl.Phys.A 904 - 905 (2013)
\bibitem{EICGSDIH} L. Zheng, E.C. Aschenauer, J.H. Lee, and B.-W. Xiao, ``Probing Gluon Saturation through Dihadron Correlations at an Electron-Ion Collider'', Phys. Rev. D 89, 074037 (2014)
\bibitem{IntroShadow} J Jalilian-Marian and X.N. Wang, ``Small x gluons in nuclei and hadrons'', Phys. Rev. D 60, 054016 (1999)
\bibitem{DenseQCD} V.P. Gon�alves ``QCD at high parton density'', Braz. J. Phys.  34 (2004)
\bibitem{ExShadow} F. Arleo and T. Gousset, ``Measuring gluon shadowing with prompt photons at RHIC and LHC'', Phys. Lett. B 660 (2008)
\bibitem{StatMechHadron} P. Huovinen and P. Petreczky , ``QCD Equation of State and Hadron Resonance Gas'', Nucl. Phys. A 837 (2010) 
\bibitem{EOSHadron} N. Sarkar and P. Ghosh , ``van der Waals hadron resonance gas and QCD phase diagram'', Phys. Rev. C 98, 014907 (2018) 
\bibitem{NuclearShadowing} Jamal. Jalilian-Marian and X.N. Wang, ``Shadowing of gluons in perturbative QCD: A comparison of different models'', Phys. Rev. D63, 096001 (2001)
\bibitem{StrongNuclear} E. Epelbaum, H.-W. Hammer, and U.G. Mei{\ss}ner, ``Modern theory of nuclear forces'', Rev. Mod. Phys. 81 (2009)
\bibitem{QGPCosmology} J. Rafelski, "Connecting QGP-Heavy Ion Physics to the Early Universe``, Nucl. Phys. B Proc. Suppl. 243-244 (2013)  
\bibitem{MITBag} S. M. Sanches Jr., F. S. Navarra, and D. A. Foga�a, ``The quark gluon plasma equation of state and the expansion of the early Universe'', Nucl. Phy. A 937 (2015)
\bibitem{QGPEOSRef} E.S. Fraga and A. Kurkela, ``Interacting quark matter equation of state for compact stars'', Astrophys. J. Lett. 781, L25 (2014)
\bibitem{ColorSuperconductor} M. G. Alford, K. Rajagopal, T. Schaefer, A. Schmitt ``Color superconductivity in dense quark matter'', Rev. Mod. Phys. 80 (2008)
%\bibitem{ColorSuperconductor} D.K. Hong, ``Aspects of color superconductivity'', Acta Phys. Polon. B 32 (2001)
\bibitem{CSCOccurrence} M. G. Alford, ``Color superconducting quark matter'', Ann. Rev. Nucl. Part. Sci. 51 (2001)

\bibitem{QCDFirstOrder} K. Rajagopal, ``Mapping the QCD phase diagram'', Nucl. Phys. A 661 (1999) 

\bibitem{EOSPhase} G. Odyniec on behalf of STAR Collaboration, ``Beam Energy Scan Program at RHIC (BES I and BES II) -- Probing QCD Phase Diagram with Heavy-Ion Collisions'', PoS CORFU2018 (2019) 


\bibitem{CriticalPointTH} Z. Fodor and S.D. Katz, ``Critical point of QCD at finite T and mu, lattice results for physical quark masses'', JHEP 04 050 (2004)

\bibitem{CriticalPointEX} S. Gupta, X. Luo, B. Mohanty, H. G. Ritter, N. Xu, ``Scale for the Phase Diagram of Quantum Chromodynamics'', Science 332 (2011)
\bibitem{LittleBang} U. Heinz, ``The Little Bang: Searching for quark-gluon matter in relativistic heavy-ion collisions'', Nucl. Phys. A 685, 414-431, 2001

\bibitem{SQMReview} R.X. Xu, ``Strange quark stars - A Review'', IAU Symp. 214 (2003)

\bibitem{SS1} Y.-Z. Fan, Y.-W. Yu, D. Xu, Z.-P. Jin, X.-F. Wu, D.-M. Wei, and B. Zhang, ``A supra-massive magnetar central engine for short GRB 130603B'', Astrophys. J. Lett. 779 (2013) 
\bibitem{SS2} Z. G. Dai, S. Q. Wang, J. S. Wang, L. J. Wang, and Y. W. Yu, ``The Most Luminous Supernova ASASSN-15lh: Signature of a Newborn Rapidly-Rotating Strange Quark Star'', Astrophys. J. 817 (2016)
\bibitem{SS3} 

\bibitem{RHICReport} D. Trbojevic and S. Peggs, ``Required Accuracy of the RHIC Circumference'', United States: N. p., Web. doi:10.2172/1119398 (1993)

\bibitem{AuStripping} M. J. Rhoades-Brown, ``The Heavy Ion Injection Scheme for RHIC'', Proc. of the Workshop on the RHIC Performance (1988)
\bibitem{FirstAuSource} D. B. Steski, J. Alessi, J. Benjamin, C. Carlson, M. Manni, P. Thieberger, and M. Wiplich, ``Operation of the Relativistic Heavy Ion Collider $Au^-$ ion source'', Review of Scientific Instruments 73, 797 (2002) 

\bibitem{RHICStrpDetail} D.B. Steski and P. Thieberger, ``Stripping foils at RHIC'', Nucl. Instrum. Meth. A 613 (2010) 
\bibitem{AuStripRef} P. Thieberger, L. Ahrens, J. Alessi, J. Benjamin, M. Blaskiewicz, J. M. Brennan, K. Brown, C. Carlson, C. Gardner, W. Fischer, D. Gassner, J. Glenn, W. Mac Kay, G. Marr, T. Roser, K. Smith, L. Snydstrup, D. Steski, D. Trbojevic, N. Tsoupas, V. Zajic, and K. Zeno, ``Improved gold ion stripping at 0.1 and 10 GeV/nucleon for the Relativistic Heavy Ion Collider'', Phys. Rev. ST Accel. Beams 11, 011001 (2008)
\bibitem{LHCReport} L. Evans, ``The Large Hadron Collider'', Phil. Trans. R. Soc. A 370 (2012) 
\bibitem{OORun} J. Brewer, A. Mazeliauskas, and W. van der Schee, ``Opportunities of OO and pO collisions at the LHC'', CERN Theory Report: CERN-TH-2021-028 (2021) 
\bibitem{LHCStrip} M. Schaumann, R. Alemany-Fernandez, H. Bartosik, T. Bohl, R. Bruce, G-H Hemelsoet, S. Hirlaender, J. Jowett, V. Kain, M. Krasny, J. Molson, G. Papotti, M.S. Camillocci, H. Timko, and J. Wenninger, ``First partially stripped ions in the LHC (${}^{208}Pb^{81+}$)'' J. Phys. Conf. Ser. 1350, 012071 (2019)

\bibitem{CYWong} C.Y. Wong, ``Introduction to high-energy heavy ion collisions'', Singapore, Singapore: World Scientific (1994) 516 p

\bibitem{IPHICText} Z.-T. Liang and X.-N. Wang , ``Globally Polarized Quark-Gluon Plasma in Noncentral A + A Collisions'', Phys.Rev.Lett. 96, 039901 (2006)
\bibitem{GuntherV3} B.Alver and G.Roland, ``Collision geometry fluctuations and triangular flow in heavy-ion collisions'', Phys. Rev. C 81, 054905 (2010) 
\bibitem{CentDef} I. Altsybeev and V. Kovalenko, ``Classifiers for centrality determination in proton-nucleus and nucleus-nucleus collisions'', EPJ Web Conf. 137, 11001
\bibitem{ALICEZDC} P. Cortese, ``Performance of the ALICE Zero Degree Calorimeters and upgrade strategy'', J. Phys. Conf. Ser. 1162, 012006 (2019)
\bibitem{CMSZDC} Oliver Suranyi, ``Study of Very Forward Neutrons with the CMS Zero Degree Calorimeter'', Universe 5 10, 210 (2019)
\bibitem{ATLASZDC} P. Dmitrieva and I. Pshenichnov, ``On the performance of Zero Degree Calorimeters in detecting multinucleon events'', Nucl. Instrum. Meth. A 906 (2018)
\bibitem{CentPlot} M. L. Miller, K. Reygers, S. J. Sanders and P. Steinberg, ``Glauber modeling in high energy nuclear collisions'', Ann. Rev. Nucl. Part. Sci. 57, 205 (2007) 
\bibitem{Glauber} R. J. Glauber, ``Quantum Optics and Heavy Ion Physics'', Nucl. Phys. A 774 (2006)
\bibitem{Optical1} J. Chauvin, D. Bebrun, A. Lounis, and M. Buenerd, ``Low and intermediate energy nucleus-nucleus elastic scattering and the optical limit of Glauber theory'', Phys. Rev. C. 28, 1970 (1983)
\bibitem{Optical2} T. Wibig and D. Sobczynska, ``Proton-nucleus cross section at high energies'', J. Phys. G: Nucl. Part. Phys. 24, 2037 (1998)
\bibitem{NPartScaling} B. B. Back, ``Studies of multiplicity in relativistic heavy-ion collisions'', 	J.Phys.Conf.Ser. 5 (2000)
\bibitem{NCollScaling} A. Milov, ``Electroweak probes with ATLAS'', PoS High-pT2017 016 (2019)
\bibitem{LeonQGP} L. Van Hove, ``Theoretical prediction of a new state of matter, the "quark-gluon plasma" (also called "quark matter")'', Part of Multipartcle Dynamics. Proceedings, 17th International Symposium, Seewinkel, Austria, June 16-20, 801-818 (1986)
\bibitem{QGPSignature} S. A. Bass, M. Gyulassy, H. Stoecker, and W. Greiner, ``Signatures of Quark-Gluon-Plasma formation in high energy heavy-ion collisions: A critical review'',  J. Phys. G 25 R1-R57 (1999)
\bibitem{QuarkoniaV} C. Quigg and J. L. Rosner, ``Quantum Mechanics with Applications to Quarkonium'', Phys. Rept. 56 167-235 (1979) 
\bibitem{QCDString} P. Petreczky, ``Quarkonium in Hot Medium'', J. Phys. G 37, 094009 (2010)
\bibitem{CSBQQ}  G. S. Bali, H. Neff, T. Duessel, T. Lippert, K. Schilling, ``Observation of string breaking in QCD'', Phys. Rev. D 71, 114513 (2005)
\bibitem{QQMelt} P. Petreczky, ``Quarkonium in Hot Medium'', J. Phys. G 37, 094009 (2010)

\bibitem{STARJpsi} STAR Collaboration, ``Measurement of inclusive $J/\psi$ suppression in Au+Au collisions at $\sqrt{s_{NN}}$ = 200 GeV through the dimuon channel at STAR'', Phys. Lett. B 797, 134917 (2019)
%\bibitem{JPsiRegen} A. Andronic, F. Arleo, R. Arnaldi, A. Beraudo, E. Bruna, D. Caffarri, Z. Conesa del Valle, J.G. Contreras, T. Dahms, A. Dainese, M. Djordjevic, E.G. Ferreiro, H. Fujii, P.B. Gossiaux, R. Granier de Cassagnac, C. Hadjidakis, M. He, H. van Hees, W.A. Horowitz, R. Kolevatov, B.Z. Kopeliovich, J.P. Lansberg, M.P. Lombardo, C. Lourenco, G. Martinez-Garcia, L. Massacrier, C. Mironov, A. Mischke, M. Nahrgang, M. Nguyen, J. Nystrand, S. Peigne, S. Porteboeuf-Houssais, I.K. Potashnikova, A. Rakotozafindrabe, R. Rapp, P. Robbe, M. Rosati, P. Rosnet, H. Satz, R. Schicker, I. Schienbein, I. Schmidt, E. Scomparin, R. Sharma, J. Stachel, D. Stocco, M. Strickland, R. Tieulent, B.A. Trzeciak, J. Uphoff, I. Vitev, R. Vogt, K. Watanabe, H. Woehri, P. Zhuang, ``Heavy-flavour and quarkonium production in the LHC era: from proton-proton to heavy-ion collisions'', Eur.Phys.J.C 76, 107 (2016)

\bibitem{JPsiRegen} A. Andronic et. al., ``Heavy-flavour and quarkonium production in the LHC era: from proton-proton to heavy-ion collisions'', Eur.Phys.J.C 76, 107 (2016)


\bibitem{STARUpsilonRef} STAR Collaboration, ``Suppression of $\Upsilon$ production in d+Au and Au+Au collisions at $\sqrt{s_{NN}}$ = 200 GeV'', Phys. Lett. B 735, 127-137 (2014)


\bibitem{CMSUpsilonRef} CMS Collaboration, ``Suppression of $\Upsilon (1S)$, $\Upsilon (2S)$, and $\Upsilon (3S)$ production in PbPb collisions at $\sqrt{s_{NN}}$ = 200 GeV'', Phys. Lett. B 770 357-379 (2017)

\bibitem{HERAJET} ZEUS Collaboration, ``Forward jet production in deep inelastic ep scattering and low-x parton dynamics at HERA'',  Phys. Lett. B 632 13-26 (2006)
\bibitem{STARJetRef} STAR Collaboration, ``Disappearance of back-to-back high $p_{T}$ hadron correlations in central Au+Au collisions at '$\sqrt{s_{NN}}$ = 200 GeV', Phys.Rev.Lett. 90, 082302 (2003)
\bibitem{ALICEJetRef} ALICE Collaboration, ``Measurements of inclusive jet spectra in pp and central Pb-Pb collisions at $\sqrt{s_{NN}}$ = 5.02 TeV'', Phys. Rev. C 101, 034911 (2020)
\bibitem{V1Tilted} P. Bozek and I. Wyskiel, ``Directed flow in ultrarelativistic heavy-ion collisions'', Phys. Rev. C 81, 054902 (2010) 
\bibitem{V1CME} CMS Collaboration, ``Constraints on the chiral magnetic effect using charge-dependent azimuthal correlations in pPb and PbPb collisions at the CERN Large Hadron Collider'', Phys. Rev. C 97, 044912 (2018)

\bibitem{EllipticFlow} A. M. Poskanzer and S.A. Voloshin, ``Methods for analyzing anisotropic flow in relativistic nuclear collisions'', Phys.Rev.C 58 1671-1678 (1998)
\bibitem{V2Eccent} R. S. Bhalerao, J.-Y. Ollitrault,``Eccentricity fluctuations and elliptic flow at RHIC'', Phys. Lett. B 641, 260-264 (2006)
\bibitem{V2STAR} STAR Collaboration, ``Elliptic flow in Au+Au collisions at $\sqrt{s_{NN}} = $ 130 GeV'', Phys. Rev. Lett. 86, 402-407 (2001)
\bibitem{V2ALICE} ALICE Collaboration, ``Elliptic flow of charged particles in Pb-Pb collisions at 2.76 TeV'', Phys. Rev. Lett. 105, 252302 (2010)
\bibitem{Hydro} 
\bibitem{SSEnhance} J. Rafelski and B. Muller, ``Strangeness Production in the Quark-Gluon Plasma'', Phys. Rev. Lett. 48, 1066 (1982)

\bibitem{StrangeSTAR} STAR Collaboration, ``Measurements of $\phi$ meson production in relativistic heavy-ion collisions at RHIC'',  Phys. Rev. C 79, 064903  (2009)
\bibitem{StrangeALICE} ALICE Collaboration, ``Enhanced production of multi-strange hadrons in high-multiplicity proton-proton collisions'', Nature Phys. 13, 535-539 (2017) 
\bibitem{SPSQGP} U. W. Heinz and M. Jacob, ``Evidence for a new state of matter: An Assessment of the results from the CERN lead beam program'', CERN Special Seminar Report, (2000)
\bibitem{BRAHMS} ``Quark Gluon Plasma an Color Glass Condensate at RHIC? The perspective from the BRAHMS experiment'' , Nucl. Phys. A 757, 1-27 (2005)
\bibitem{PHOBOS} PHOBOS Collaboration, ``The PHOBOS Perspective on Discoveries at RHIC'', Nucl. Phys. A 757, 28-101 (2005)
\bibitem{STAR} STAR Collaboration, ``Experimental and theoretical challenges in the search for the quark?gluon plasma: The STAR Collaboration's critical assessment of the evidence from RHIC collisions'', Nucl. Phys. A 757, 102-183 (2005)
\bibitem{PHENIX} PHENIX Collaboration, ``Formation of dense partonic matter in relativistic nucleus-nucleus collisions at RHIC: Experimental evaluation by the PHENIX collaboration'', Nucl. Phys. A 757, 184-283 (2005)
\bibitem{QGPLHC} J. Rafelski, ``Discovery of Quark-Gluon-Plasma: Strangeness Diaries'', Eur. Phys. J. ST 229, 1-140 (2020)
\bibitem{QGPLifeTime} C. Markert, R. Bellwied, and I. Vitev, ``Formation and decay of hadronic resonances in the QGP'', Phys. Lett. B 669, 92-97 (2008) 
\bibitem{QGPThermal} T. Kodama, ``Hunt for the quark-gluon plasma: 20 years later'', Braz. J. Phys. 34, 205-210 (2004)
\bibitem{QGPChemical} A. Kurkela and A. Mazeliauskas, ``Kinetic and Chemical Equilibration of Quark-Gluon Plasma'',  Springer Proc. Phys. 250, 177-181 (2020)
\bibitem{sQGP} J. L. Nagle, ``The Letter S (and the sQGP)'', Eur. Phys. J. C 49, 275-279 (2007)
\bibitem{LatticeEOS} S.M. Sanches, F.S. Navarra, and D.A. Fogaca, ``The quark gluon plasma equation of state and the expansion of the early Universe'', Nucl. Phys. A 937, 1-16  (2015) 
\bibitem{QGPEtaOverS} U. Heinz, C. Shen, and H. Song, ``The viscosity of quark-gluon plasma at RHIC and the LHC'', AIP Conf. Proc. 1441, 766-770 (2012)
\bibitem{ADSCFT} G. Policastro, D.T. Son, and A.O. Starinets, ``Shear viscosity of strongly coupled N=4 supersymmetric Yang-Mills plasma'', Phys. Rev. Lett. 87, 081601 (2001)
\bibitem{4DHydro} P. F. Kolb and U. Heinz, ``Hydrodynamic description of ultra relativistic heavy-ion collisions'', Part of Quark-gluon plasma 4, 634-714 (2003)
\bibitem{Bjorken} J. Bjorken, ``Highly Relativistic Nucleus-Nucleus Collisions: The Central Rapidity Region'', Phys.Rev.D 27 140-151 (1983) 
\bibitem{QGPGen} B. V. Jacak and B. M�ller, ``The Exploration of Hot Nuclear Matter'', Science 337, 310-314 (2012)
\bibitem{QGPOpaque} A. Adil and M. Gyulassy, ``Energy systematics of jet tomography at RHIC: $\sqrt{s_{NN}}$ = 62.4 vs. 200 AGeV'', 
\bibitem{BigQuestions} Heavy Ion Collisions: The Big Picture, and the Big Questions, ``Heavy Ion Collisions: The Big Picture, and the Big Questions'',  Ann. Rev. Nucl. Part. Sci. 68, 339-376 (2018)
\bibitem{Rutherford} E. Rutherford, ``The Scattering of $\alpha$ and $\beta$ Particles by Matter and the Structure of the Atom'', Philos. Mag, 6, 21 (1911) 
\bibitem{Henry} H. W. Kandall, ``Deep inelastic scattering: Experiments on the proton and the observation of scaling'', Rev. Mod. Phys. 63, 597-614 (1991)
\bibitem{Richard} R. E. Taylor, ``Deep inelastic scattering: The Early years'', Rev. Mod. Phys. 63, 573-595 (1991) 
\bibitem{Jerry} J. I. Friedman, ``Deep inelastic scattering: Comparisons with the quark model'', Rev. Mod. Phys. 63, 615-629 (1991)
\bibitem{HardProbes}
\bibitem{HPSeries} C. Lourenco and H. Satz, ``Proceedings, 1st International Conference on Hard and Electromagnetic Probes of High-Energy Nuclear Collisions (Hard Probes 2004) : Ericeira, Portugal, November 4-10, 2004'', Eur. Phys. J. C 43 1-4  (2005) 
\bibitem{JetPath} B. Betz and M. Gyulassy, ``Constraints on the Path-Length Dependence of Jet Quenching in Nuclear Collisions at RHIC and LHC'', JHEP 08, 090 (2014)
\bibitem{CMSJETEVENT} CMS Collaboration, ``Event displays and some infographics of jets in heavy ion collisions'', CMS-PHO-EVENTS-2021-007, \url{http://cds.cern.ch/record/2757389} 
\bibitem{ModJetSub} J. Casalderrey-Solana, G. Milhano, D. Pablos, and K. Rajagopal, ``Modification of Jet Substructure in Heavy Ion Collisions as a Probe of the Resolution Length of Quark-Gluon Plasma'', JHEP 01, 044 (2020) 
%\bibitem{PHENIXGamma} PHENIX Collaboration, ``Measurement of Direct Photons in Au+Au Collisions at $\sqrt{s_{NN}}$ = 200 GeV'', Phys. Rev. Lett. 109, 152302 (2012)
\bibitem{ALICEJETSub} ALICE Collaboration, ``Exploring jet substructure with jet shapes in ALICE'', Nucl. Phys. A 967 528-531(2017) 
\bibitem{CMSJetSub} C. McGinn, ``Mapping the redistribution of jet energy in PbPb collisions at the LHC with CMS'', MIT PHD Thesis (2019)
\bibitem{CMSGammaRef} CMS Collaboration, ``Measurement of isolated photon production in pp and PbPb collisions at  $\sqrt{s_{NN}}$ = 2.76 TeV'', Phys. Lett. B 710, 256 (2012) 
\bibitem{CMSZRef} CMS Collaboration, ``Study of Z production in PbPb and pp collisions at $\sqrt{s_{NN}}$ = 2.76 TeV in the dimuon and dielectron decay channels'', JHEP 03, 022 (2015)
\bibitem{PRLCover} CMS Collaboration, ``Study of Jet Quenching with $Z$ + jet Correlations in Pb-Pb and pp collisions at  $\sqrt{s_{NN}}$ = 5.02 TeV'', Phys. Rev. Lett. 119, 082301 (2017) 
\bibitem{FONLLRef} 
\bibitem{QCDFFunc} A. Metz and A. Vossen, ``Parton Fragmentation Functions'', Prog. Part. Nucl. Phys. 91, 136-202  (2016)
\bibitem{HadronScale} S.J. Brodsky, H.J. Pirner, and J. Raufeisen ``Scaling properties of high $p_T$ inclusive hadron production'', Phys. Lett. B 637, 58-63 (2006)
\bibitem{LHCbFF} LHCb Collaboration, ``Measurement of $f_s/f_u$ variation with proton-proton collision energy and B-meson kinematics'', Phys. Rev. Lett. 124, 122002 (2020)
\bibitem{GMISQM} ALICE Collaboration ``Charm-quark fragmentation fractions and production cross section at midrapidity in pp collisions at the LHC'',  CERN-EP-2021-088

\bibitem{HQReview} X. Dong, Y.-J. Lee, and R. Ralf, ``Open Heavy-Flavor Production in Heavy-Ion Collisions'',  Ann. Rev. Nucl. Part. Sci. 69, 417-445 (2019)
\bibitem{HQTau} Y. Liu, C. M. Ko, and F. Li, ``Heavy quark correlations and the effective volume for quarkonia production'', Phys. Rev. C 93, 034901 (2016)

\bibitem{HQRaff} F. Prino and R. Rapp, ``Open Heavy Flavor in QCD Matter and in Nuclear Collisions'', J. Phys. G 43, 093002 (2016)
\bibitem{HQJamie} A. M. Adare, M. P. McCumber, J. L. Nagle, and P. Romatschke, ``Tests of the Quark-Gluon Plasma Coupling Strength at Early Times with Heavy Quarks'', Phys. Rev. C 90, 024911 (2014)
\bibitem{HQCollELoss} 
\bibitem{HQRadELoss} 
\bibitem{ADSCFTDrag} S. S. Gubser, ``Drag force in AdS/CFT'', Phys. Rev. D 74, 126005 (2006)
\bibitem{HQHoloELoss} A. Ficnar, J. Noronha, and M. Gyulassy, ``Non-conformal Holography of Heavy Quark Quenching'', Nucl. Phys. A 855 (2011)


\bibitem{HQELossFirst} Y. Akiba, ``Quest for the quark-gluon plasma - hard and electromagnetic probes'' Prog. Theor. Exp. Phys. 2015, 03A105 (2015).
\bibitem{Brems} H. Bichsel and H. Schindler, ``The Interaction of Radiation with Matter'', In: Fabjan C., Schopper H. (eds) Particle Physics Reference Library. Springer, Cham. (2020)
\bibitem{DEADCONE} Y. L. Dokshitzer and D. E. Kharzeev, ``Heavy quark colorimetry of QCD matter'', Phys. Lett. B 519, 199 (2001) 
\bibitem{qhatStudy} JET Collaboration, ``Extracting the jet transport coefficient from jet quenching in high-energy heavy-ion collisions'', Phys. Rev. C 90, 014909 (2014)
\bibitem{JetTransProbe} S.-Q.Li, W.-J. Xing, F.-L. Liu, S. Cao, and C.-Y. Qin, ``Heavy flavor quenching and flow: the roles of initial condition, pre-equilibrium evolution, and in-medium interaction'', Chin. Phys. C 44, 114101 (2020)

\bibitem{TAMUModel} M. He, R. J. Fries, and R. Rapp, ``Heavy flavor at the large hadron collider in a strong coupling approach", Phys. Lett. B 735, 445 - 450 (2014) 
\bibitem{LQCDTAMU} F. Riek and R. Rapp, ``Quarkonia and heavy-quark relaxation times in the quark-gluon plasma'', Phys. Rev. C 82, 035201 (2010).
\bibitem{RRM1} Min He, Rainer J. Fries, and Ralf Rapp, ``Heavy-quark diffusion and hadronization in quark-gluon plasma'', Phys. Rev. C 86, 014903 (2012)
\bibitem{CaoSunKo} S. Cao et. al., ``Charmed hadron chemistry in relativistic heavy-ion collisions", Phys. Lett. B 807, 135561 (2020) 
\bibitem{CaoLH1} S. Cao, G.-Y. Qin, and S. A. Bass, Phys. Rev. C 88, 044907 (2013)
\bibitem{CaoLH2} S. Cao, G.-Y. Qin, and S. A. Bass, Phys. Rev. C 92, 024907 (2015)
\bibitem{RRM2} Min He and Ralf Rapp, ``Hadronization and Charm-Hadron Ratios in Heavy-Ion Collisions'', Phys. Rev. Lett. 124, 042301 (2020)
\bibitem{PYTHIAFrag} T. Sjostrand, S. Mrenna, and P. Z. Skands, ``PYTHIA 6.4 Physics and Manual'', JHEP 0605, 026 (2006)
\bibitem{StrangetoLight} I. Kuznetsova and J. Rafelski, ``Heavy flavor hadrons in statistical hadronization of strangeness-rich QGP'', Eur. Phys. J. C 51, 113-133 (2007)
\bibitem{BaryontoMeson} Y. Oh, C. M. Ko, S. H. Lee, and S. Yasui, ``Heavy baryon/meson ratios in relativistic heavy-ion collisions'', Phys. Rev. C 79, 044905 (2009) 
\bibitem{NCDScaling} Z. Tang, L. Yi, L. Ruan, M. Shao, H. Chen, C. Li, B. Mohanty, P. Sorensen, A. Tang, and Z. Xu, ``Statistical Origin of Constituent-Quark Scaling in the QGP hadronization'', Chin. Phys. Lett. 30 031201(2013) 
\bibitem{STARD0v2} STAR Collaboration, ``Measurement of $D^0$ Azimuthal Anisotropy at Midrapidity in Au + Au Collisions at '$\sqrt s_{NN} =$ 200 GeV', Phys. Rev. Lett. 118, 212301 (2017)
\bibitem{CMSD0v2} CMS Collaboration, ``Elliptic Flow of Charm and Strange Hadrons in High-Multiplicity $p + Pb$ Collisions at $\sqrt s_{NN} = $ 8.16 TeV'', Phys. Rev. Lett. 121, 082301 (2018)
\bibitem{STARD0RAA} STAR Collaboration, ``Observation of $D^0$ Meson Nuclear Modifications in Au+Au Collisions at $\sqrt s_{NN} =$ 200 GeV'', Phys. Rev. Lett. 113, 142301 (2014) 
\bibitem{CMSD0RAA} CMS Collaboration, ``Nuclear modification factor of $D^0$ mesons in PbPb collisions at $\sqrt s_{NN} =$ 5.02 TeV'', Phys. Lett. B 782, 474-496 (2018) 
\bibitem{STARLambdaCD0} STAR Collaboration, ``First measurement of $\Lambda_C$ baryon production in AuAu collisions at $\sqrt{s_{NN}} =$ 200 GeV'', Phys. Rev. Lett. 124, 172301 (2020) 
\bibitem{ALICELambdaCD0} ALICE Collaboration, ``$\$\Lambda_C^+$ production in Pb - Pb collisions at $\sqrt s_{NN} =$ 5.02 TeV'', Phys.Lett.B 793, 212-223 (2019)
\bibitem{DDbar} ALICE Correlation, ``Measurement of azimuthal correlations between D mesons and charged hadrons with ALICE at the LHC'', EPJ Web Conf. 80, 00034 (2014)
\bibitem{DJet} CMS Collaboration, ``Studies of charm quark diffusion inside jets using PbPb and pp collisions at $\sqrt s_{NN} =$ 5.02 TeV'', Phys. Rev. Lett. 125, 102001 (2020) 





%Chapter 2
\bibitem{CMSDetector}  CMS Collaboration, ``The CMS experiment at the CERN LHC'', JINST 3 S08004 (2008).
\bibitem{HiggsCMS} CMS Collaboration, ``A New Boson with a Mass of 125 GeV Observed with the CMS Experiment at the Large Hadron Collider'', Science 338, 1569-1575 (2012) 
\bibitem{CMSTrigger} CMS Collaboration, ``The CMS trigger system'', JINST 12 01, P01020 (2017)
\bibitem{MBTrigger} Y. Chao, ``Minimum-Bias and Underlying Event Studies at CMS'', Proceedings, 28th International Conference on Physics in Collision (PIC 2008) : Perugia, Italy, June 25-28, (2008)
\bibitem{BAnaDimuonTrigger} Zhaozhong Shi, ``Measurement of $B^0_s$ and $B^+$ meson yields in PbPb collisions at $\sqrt{s_{NN}} = $ 5.02 TeV'', CMS-PAS-HIN-19-011
\bibitem{CMSSilicon} CMS Collaboration, ``The Phase-2 Upgrade of the CMS Tracker'', CERN-LHCC-2017-009
\bibitem{CMSTrackComp} CMS Collaboration, ``Description and performance of track and primary-vertex reconstruction with the CMS tracker'', JINST 9 10, P10009 (2014)
\bibitem{CMSTrack1} P. Billoir, ``Progressive track recognition with a Kalman like fitting procedure'', Comput. Phys. Commun. 57, 390 (1989)  
\bibitem{CMSTrack2} P. Billoir and S. Qian, ``Simultaneous pattern recognition and track fitting by the Kalman filtering method'', Nucl. Instrum. Meth. A 294, 219 (1990) 
\bibitem{CMSTrack3} R. Mankel, ``A Concurrent track evolution algorithm for pattern recognition in the HERA-B main tracking system'', Nucl. Instrum. Meth. A 395, 169 (1997) 
\bibitem{Kalman} R. Fruhwirth, ``Application of Kalman filtering to track and vertex fitting'', Nucl. Instrum. Meth. A 262, 444 (1987) 
\bibitem{DAAlgo} K. Rose, ``Deterministic Annealing for Clustering, Compression, Classification, Regression and related Optimisation Problems'', Proceedings of the IEEE 86 (1998)
\bibitem{ECALReso} CMS Collaboration, ``The CMS ECAL performance with examples'', JINST 9 C02008 (2014)
\bibitem{HCALReport} CMS Collaboration, ``The CMS hadron calorimeter project : Technical Design Report'', CERN-LHCC-97-031 (1997)
\bibitem{HFInfo} A. Penzo and Y. Onel, ``The CMS-HF quartz fiber calorimeters'', J.Phys.Conf.Ser. 160, 012014 (2009)
\bibitem{HFCentRef} CMS Collaboration, ``Observation and studies of jet quenching in PbPb collisions at nucleon-nucleon center-of-mass energy = 2.76 TeV'', Phys. Rev. C 84, 024906 (2011)
\bibitem{CASZDCRef} B. Roland, ``Forward Physics Capabilities of CMS with the CASTOR and ZDC detectors'', Part of Proceedings, 17th International Workshop on Deep-Inelastic Scattering and Related Subjects (DIS 2019), Madrid, Spain, April 26-30 (2009)


%Chapter 3

\bibitem{CMSDAQ} T. Bawej el. at., ``The New CMS DAQ System for Run-2 of the LHC'', IEEE Trans. Nucl. Sci. 62, 1099-1103 (2015) 
\bibitem{CMSPIXInfo} CMS Collaboration, ``Commissioning and performance of the CMS pixel tracker with cosmic ray muons'', JINST 5 T03007 (2010)
\bibitem{GSF} W. Adam, R. Fruhwirth, A. Strandlie, and T. Todorov, ``Reconstruction of electrons with the Gaussian-sum filter in the CMS tracker at LHC'', J. Phys. G 31 N9 (2005) 
\bibitem{AVT} R. Fruhwirth, W. Waltenberger, and P. Vanlaer, ``Adaptive vertex fitting'', J. Phys. G 34 N343 (2007) 






%Chapter 4
\bibitem{PYTHIA2} R. Field, ``Early LHC Underlying Event Data - Findings and Surprises'', in Hadron collider physics. Proceedings, 22nd Conference, HCP 2010, Toronto, Canada, August 23-27 (2010)
\bibitem{EvtGen} D. Lange, ``The EvtGen particle decay simulation package'', Nucl. Instrum. Meth. A 462  (2001)
\bibitem{PHOTOS} E. Barberio, B. van Eijk, and Z.Was, ``PHOTOS: A Universal Monte Carlo for QED radiative corrections in decays'', Comput. Phys. Commun. 66 115 -128 (1991) 
\bibitem{HYDJET} I. P. Lokhtin and A. M. Snigirev, ``A model of jet quenching in ultrarelativistic heavy ion collisions and high-pT hadron spectra at RHIC'', Eur. Phys. J. C45  211(2006)
\bibitem{EvtSel} CMS Collaboration, ``Transverse momentum and pseudorapidity distributions of charged hadrons in pp collisions at $\sqrt{s}$ = 0.9 and 2.76 TeV'', JHEP 02 041 (2010)
\bibitem{SoftMuon} CMS Collaboration, ``Performance of CMS muon reconstruction in pp collision events at $\sqrt{s}$ = 7 TeV'', JINST 7 P10002 (2012)
\bibitem{MuonAccRef} CMS Collaboration, ``Muon performance studies in 2017 pp and 2018 PbPb 5.02TeV data'', CMS Analysis Note: AN-18-316 (2020)
\bibitem{PbPbMulti} 
\bibitem{MVARef} 
\bibitem{KSTest} V. W. Berger and Y. Zhou, "Kolmogorov?Smirnov test: Overview", Wiley Statsref: Statistics Reference Online, (2014)
\bibitem{Boosting} M. Kearns and L. Valiant (1989). Crytographic limitations on learning Boolean formulae and finite automata. Symposium on Theory of Computing. 21. ACM. 433 - 444. (1989)
\bibitem{TMVA} J. Therhaag, ``TMVA: Toolkit for multivariate data analysis'', AIP Conf.Proc. 1504, 1013-1016 (2012)
\bibitem{CMSBPH} CMS Collaboration ``Measurement of the $B^+$ hadron production cross section in pp collisions at 13 TeV'', CMS Physics Analysis Summary, CMS-PAS-BPH-15-004 (2015)
\bibitem{ROOFIT} Wouter Verkerke and David P. Kirkby, ``The RooFit toolkit for data modeling'', eConf. C0303241, MOLT007 (2003) 
\bibitem{TnPMethod} CMS Collaboration, ``Upsilon production cross-section in pp collisions at $\sqrt{s}$  = 7 TeV'', Phys. Rev. D 83, 112004 (2011) 




























\end{thebibliography}  



\bibliographystyle{unsrturl}
\bibliography{refs}
