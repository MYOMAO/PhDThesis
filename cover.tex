% -*-latex-*-
% 
% For questions, comments, concerns or complaints:
% thesis@mit.edu
% 
%
% $Log: cover.tex,v $
% Revision 1.9  2019/08/06 14:18:15  cmalin
% Replaced sample content with non-specific text.
%
% Revision 1.8  2008/05/13 15:02:15  jdreed
% Degree month is June, not May.  Added note about prevdegrees.
% Arthur Smith's title updated
%
% Revision 1.7  2001/02/08 18:53:16  boojum
% changed some \newpages to \cleardoublepages
%
% Revision 1.6  1999/10/21 14:49:31  boojum
% changed comment referring to documentstyle
%
% Revision 1.5  1999/10/21 14:39:04  boojum
% *** empty log message ***
%
% Revision 1.4  1997/04/18  17:54:10  othomas
% added page numbers on abstract and cover, and made 1 abstract
% page the default rather than 2.  (anne hunter tells me this
% is the new institute standard.)
%
% Revision 1.4  1997/04/18  17:54:10  othomas
% added page numbers on abstract and cover, and made 1 abstract
% page the default rather than 2.  (anne hunter tells me this
% is the new institute standard.)
%
% Revision 1.3  93/05/17  17:06:29  starflt
% Added acknowledgements section (suggested by tompalka)
% 
% Revision 1.2  92/04/22  13:13:13  epeisach
% Fixes for 1991 course 6 requirements
% Phrase "and to grant others the right to do so" has been added to 
% permission clause
% Second copy of abstract is not counted as separate pages so numbering works
% out
% 
% Revision 1.1  92/04/22  13:08:20  epeisach

% NOTE:
% These templates make an effort to conform to the MIT Thesis specifications,
% however the specifications can change. We recommend that you verify the
% layout of your title page with your thesis advisor and/or the MIT 
% Libraries before printing your final copy.
\title{Analysis of Beauty Quark Hadronization in Vacuum and Quark-Gluon Plasma with CMS}

\author{Zhaozhong Shi}

%\author{B. A., in Physics, University of California, Berkeley (2016)}

\prevdegrees{B.A., University of California, Berkeley (2016)}

% If you wish to list your previous degrees on the cover page, use the 
% previous degrees command:
%       \prevdegrees{A.A., Harvard University (1985)}
% You can use the \\ command to list multiple previous degrees
%       \prevdegrees{B.S., University of California (1978) \\
%                    S.M., Massachusetts Institute of Technology (1981)}
\department{Department of Physics}

% If the thesis is for two degrees simultaneously, list them both
% separated by \and like this:
% \degree{Doctor of Philosophy \and Master of Science}
\degree{Doctor of Philosophy in Physics}

% As of the 2007-08 academic year, valid degree months are September, 
% February, or June.  The default is June.
\degreemonth{September}
\degreeyear{2021}
\thesisdate{September 5, 2021}

%% By default, the thesis will be copyrighted to MIT.  If you need to copyright
%% the thesis to yourself, just specify the `vi' documentclass option.  If for
%% some reason you want to exactly specify the copyright notice text, you can
%% use the \copyrightnoticetext command.  
%\copyrightnoticetext{\copyright IBM, 1990.  Do not open till Xmas.}

% If there is more than one supervisor, use the \supervisor command
% once for each.
\supervisor{Yen-Jie Lee}{Associate Professor}

% This is the department committee chairman, not the thesis committee
% chairman.  You should replace this with your Department's Committee
% Chairman.
\chairman{Nergis Mavalvala}{Associate Department Head of Physics}

% Make the titlepage based on the above information.  If you need
% something special and can't use the standard form, you can specify
% the exact text of the titlepage yourself.  Put it in a titlepage
% environment and leave blank lines where you want vertical space.
% The spaces will be adjusted to fill the entire page.  The dotted
% lines for the signatures are made with the \signature command.
\maketitle

% The abstractpage environment sets up everything on the page except
% the text itself.  The title and other header material are put at the
% top of the page, and the supervisors are listed at the bottom.  A
% new page is begun both before and after.  Of course, an abstract may
% be more than one page itself.  If you need more control over the
% format of the page, you can use the abstract environment, which puts
% the word "Abstract" at the beginning and single spaces its text.

%% You can either \input (*not* \include) your abstract file, or you can put
%% the text of the abstract directly between the \begin{abstractpage} and
%% \end{abstractpage} commands.

% First copy: start a new page, and save the page number.
\cleardoublepage
% Uncomment the next line if you do NOT want a page number on your
% abstract and acknowledgments pages.
% \pagestyle{empty}
\setcounter{savepage}{\thepage}
\begin{abstractpage}
% $Log: abstract.tex,v $
% Revision 1.1  93/05/14  14:56:25  starflt
% Initial revision
% 
% Revision 1.1  90/05/04  10:41:01  lwvanels
% Initial revision
% 
%
%% The text of your abstract and nothing else (other than comments) goes here.
%% It will be single-spaced and the rest of the text that is supposed to go on
%% the abstract page will be generated by the abstractpage environment.  This
%% file should be \input (not \include 'd) from cover.tex.
An analysis of fully reconstructed $B^0_s$ and $B^+$ mesons decay into $J/\psi$ and strange hadrons using Compact Muon Solenoid (CMS) Experiment 2017 pp dataset and 2018 PbPb data at the center of mass energy per nucleon $\sqrt{s_{NN}} = 5.02$ TeV at the Large Hadron Collider (LHC) is presented. We apply machine learning techniques along with multivariate analysis to obtain significant B-meson signals and extend the kinematic regime of B-meson measurements with higher precision. In our analysis, $B^0_s$ signal of greater than 5 $\sigma$ significance is observed for the first time in heavy-ion collisions. The measured $B^0_s$/$B^+$ ratio in PbPb along with pp references are compared with theoretical model predictions. These results will help elucidate the beauty quark hadronizaton mechanisms in vacuum and quark-gluon plasma at the LHC energy. Significant B-meson signals have also been observed at low very $p_T$ and high multiplicity in pp collisions, which will allow us study beauty hadrochemistry in small systems and energy loss mechanism in the future. 
\end{abstractpage}

% Additional copy: start a new page, and reset the page number.  This way,
% the second copy of the abstract is not counted as separate pages.
% Uncomment the next 6 lines if you need two copies of the abstract
% page.
% \setcounter{page}{\thesavepage}
% \begin{abstractpage}
% % $Log: abstract.tex,v $
% Revision 1.1  93/05/14  14:56:25  starflt
% Initial revision
% 
% Revision 1.1  90/05/04  10:41:01  lwvanels
% Initial revision
% 
%
%% The text of your abstract and nothing else (other than comments) goes here.
%% It will be single-spaced and the rest of the text that is supposed to go on
%% the abstract page will be generated by the abstractpage environment.  This
%% file should be \input (not \include 'd) from cover.tex.
An analysis of fully reconstructed $B^0_s$ and $B^+$ mesons decay into $J/\psi$ and strange hadrons using Compact Muon Solenoid (CMS) Experiment 2017 pp dataset and 2018 PbPb data at the center of mass energy per nucleon $\sqrt{s_{NN}} = 5.02$ TeV at the Large Hadron Collider (LHC) is presented. We apply machine learning techniques along with multivariate analysis to obtain significant B-meson signals and extend the kinematic regime of B-meson measurements with higher precision. In our analysis, $B^0_s$ signal of greater than 5 $\sigma$ significance is observed for the first time in heavy-ion collisions. The measured $B^0_s$/$B^+$ ratio in PbPb along with pp references are compared with theoretical model predictions. These results will help elucidate the beauty quark hadronizaton mechanisms in vacuum and quark-gluon plasma at the LHC energy. Significant B-meson signals have also been observed at low very $p_T$ and high multiplicity in pp collisions, which will allow us study beauty hadrochemistry in small systems and energy loss mechanism in the future. 
% \end{abstractpage}

\cleardoublepage

\section*{Acknowledgments}

%This is the acknowledgements section. You should replace this with your own acknowledgements.

Today, sitting in front of my Macbook and writing up my PhD thesis, I am recalling the nice memory over the past 5 years of my academic and research journey at MIT. Summarizing the work I have done. I have overcome countless challenges and seized every opportunity. I have absorbed plenty of knowledge like nutrients and lead work alcoholic life. I used to work more than many hours, particularly when deadlines are approaching. I am glad that I have overcome countless of deadlines and managed to finish most of my tasks on time during my graduate studies. Because of my hard work and achievements, During my graduate studies at MIT, I have been awarded the NSF-GRFP and DOE SCGSR fellowships. While I am proud of my academic and research accomplishments, the more importance things are love and friendship. I have been to many memorable places around the world and met many interesting people. They have taught me more than I can learn from papers, lectures, and experiments. I owed them a lot and am indebted to everyone for their influences on me. Without them, I will not be able to finish this work and reach my current maturity.

On the cloudy afternoon of February 12, 2016, I was anxiously waiting for my graduate admission decisions. At around 2pm when I am returning home from Berkeley, I received my acceptance to MIT from the MIT Department of Physics Coordinator Catherine Modica. At that moment, I was still on my BART from UC Berkeley back to San Francisco. I was very excited and immediately called my family to share this great news. I understood that this would be the start of my exciting career as a PhD candidate in Nuclear Physics research. Without thinking too much, I quickly accepted the offer and got ready to join MIT.

During the MIT Open House, I met a lot of amazing people. I first visited the MIT Heavy Ion Group and then made friends with many of my peers. I would really like to thank Constantin Weisser for his generous accommodation during the Open House. I met my future PhD advisor Professor Yen-Jie Lee and LNS director Bolek Wyslouch. I also met my fellow students at LNS such as Lauren Yates, Sangbaek Lee, and Yi Jia. They are all very nice and kind. We have spent our first three semesters together taking classes and discussing physics. Particularly, I want to mention Sangbaek that we used to discuss a lot about our life stories and future perspectives together. It was a quite memorable experience. I also frequently visited lattice QCD theorist Anthony Grebe. I still remembered the time we discussed physics questions, debated ideas, derive formulae at the Center for Theoretical Physics. He is a truly humble and kind person. I wish him to have a successful future career in theoretical particle physics.


As a resident of MIT Tang Hall, I met many interesting people. Here, I would like to thank Ali Fahimniya who also lived at Tang Hall for his first year. We often chat with each other during the cookies social in the Pappalardo Room. We shared our life experience and future plan. It turns out that we defended our PhD just a day apart from each other. 

Shortly after attending MIT, I decided to join the MIT Heavy Ion Group and pursued my scientific research career in Experimental High Energy Nuclear Physics as my PhD thesis topic for the next 5 years. Members of the MIT Heavy Ion Group have provided me with general supports and comprehensive guidance towards my PhD. Without them, I would not be able to overcome challenges and finish my thesis research projects. 


At the graduate student's office, my desk is located in the northwest corner. Senior members of faculties in our group offered me excellent training and many opportunities to become an independent researcher and build up my international reputation in the field. I would particularly like to thank Professor Yen-Jie Lee for his long and enduring support, both in research and life, for the past 5 years. He has provided me with many opportunities and ideas to carry out physics analysis and fought in the CMS collaboration to allow me to give talks in internationally recognized conferences. I am also deeply indebted to him for his care of my career development and letters of recommendation for fellowships and postdoctoral scholar positions. Thanks to his nurture, I have transformed from a student knowing almost nothing to become an independent researcher with some big-picture visions. I am also grateful to the MIT Heavy Ion Group leader Professor Gunther Roland for his generous supports to allow me to work on sPHENIX and EIC projects. His letters of recommendation and grant management advice helps me a lot to get my first postdoctoral position and become professional in developing leadership skills in academia. Both of them have provided me with opportunities to work at different places around the world such as MIT, Fermilab, BNL, and CERN. Aside from them, I appreciated the general academic etiquette and soft skills from Professor Bolek Wyslouch.


My fellow senior graduate colleagues in my group have influenced me a lot. First, the most senior graduate student Dragos Velicanus showed me around about our group and guided me through the basics. He also provided a lot of remarks on my fellowship applications. Then, Alex Barberi sat next to my office desk. We have worked together and talked about funny anecdotes. I also made acquittances with Ta-Wei Wang. He created the skimmed files for me to play with and answered me many technical questions in C$++$ programming and B mesons full reconstructions. As a newbie in the CMS analysis, I used to frequently bother Jing Wang in the analysis techniques and questions on codes. However, she still patiently guided and helped me until my confusion was gone. As the CMS spectra heavy flavor group leader, she also gave a lot of feedback and suggestions to validate the analyses. Ran Bi, whose research was on photon-jet studies, was a bit reserved but still kindly helped me a lot with CMS software and technical concepts for data processing and analysis. Kaya Tatar was a very nice and resourceful person. We used to discuss many physics and technical questions. Chris McGinn was one of my best friends at the MIT Heavy Ion Group. He provided me with a lot of technical assistance and helped me debug my codes. His help was very effective. I really appreciated that. We also debated many physics concepts and politics issues. Austin Baty was one of my buddies at CERN who used to share his experience in life and studies with me.  

Junior graduate students such as Michael Peters, Molly Taylor, and Gwang-Jun Kim were all very friendly and kind. I will never forget the time when Michael and I initiated the debate and keep on our position to try to convince with each other. I also enjoyed working with Gwang-Jun and still remember the moments we stay up many days and nights fighting to get the analysis from pre-approve to approval stage. I also would like to thank Molly for helping me look after my remaining items near my desks in the MIT office since I went to CERN. 


Staff members of our group were very professional and resourceful. Dr. Gian Michele guided me through my analysis and gave me useful feedback for my first Quark Matter talk. I have learned a lot from him in the first two years of my PhD studies. Dr. Camellia Mironov has guided me through the CMS collaboration bureaucracy and went through the procedures to present and publish papers. Dr. George Stephans has provided me with many suggestions in MC simulations, technical questions, physics concepts, and computing knowledge. Dr. Christoph Roland has shared me with a lot of heavy-ion physics anecdotes, career development stories, detector hardware knowledge, software development skills, and analysis techniques. Finally, I would like to thank Dr. Ivan Cali for his suggestions in windows codes programming, software development, and product design. 

%Faculty Yen-Jie; emphasize. 

I would also like to thank my fellow students at the PPC including Dr. Dylan Hsu, Dr. Brandon Allen, Dr. Sid Narayanan, Dr. Stephanie Brandt, and Jeff Krupa. They are all my good friends. We treated each other like brothers and made fun of each other. I still remembered I visited the PPC office almost every day and shared how my day was with them. They used to say I would not be able to graduate in 6 years but I did manage to get my PhD within 5 years, which surprised them a lot. Jeff was also my buddy. We used to go out for lunch regularly and walked along the beach every semester. I missed them all very much!

Aside from that, many other professors at MIT have helped me a lot academically and professionally. I would particularly thank my academic advisor Professor Lindley Winslow. She is very kind, considerate, and helpful. She gave me very good advice on fellowship applications, courses selections, degree preparation, and postdoc tips. My graduate career has become much smoother and planned thanks to her kind suggestions. In addition, I would like to thank Professor Iain Stewart, Richard Milner, and Krishna Rajagopal for occasional discussions. I would also like to thank Professor Barton Zwiebach for serving as the reader in my PhD thesis committee and provided me with some useful feedback about my defense and presentation.

Outside MIT, I have also met many senior sages who have given me a lot of wisdom. As a member of sPHENIX calorimetry test beam crews at Fermilab, I met Dr. Craig Woody and worked with him during the first days in 2017. After that, he has been my mentor for the past 5 years and served as my laboratory scientist for the Electron-Ion Collider Electromagnetic Calorimeter project in my DOE SCGSR award. Moreover, he has written many letters for me and provided me with a lot of resources for career development. I am very thankful for his support. I also met Dr. Jin Huang and received many technical and analysis guidances for my sPHENIX EMCAL, heavy flavor simulation, and EIC EMCAL projects. I would also like to thank Dr. John Haggerty for his creative training during my shift at Fermilab in 2017 and 2018. At CERN, I felt very fortunate to work with Professor Mario Sitta on ALICE ITS project. He was my best friend during my stay in France and has kindly helped me move my personal items around with his car and debug my programs. Without him, my stay at CERN could be very miserable. I would also like to express my sincere gratitude to my collaborator Nuno Leonardo at LIP in Portugal. Collaborating on the B-meson analysis, he has provided me with many concrete solutions to improve the analysis, ideas to overcome the challenges from the CMS collaboration, and workforce supports to complete part of the projects. He has saved this analysis many times. His efforts, along with my hard work, finally managed to make it possible and become part of my thesis. Without his support, my analysis would be stuck in many steps and may not be able to advance to the current ready for submission stage. His urgent emails and slack messages were still clearly remembered in my mind. I also met Ming Liu during the Fermilab test beam shift and ALICE ITS commissioning at CERN. We worked together at that time. We will work together as a postdoc and a mentor for the next three years. Finally, I am very lucky to have a chance to have a nice lunch with Professor James Bjorken. I have really learned many stories in the great old days of the Golden Age of particle physics and about the fun of physics with him. I also got to know his recent research interests and received some career advice from him. It was my great honor to chat with this great mind.

In addition to my mentors, colleagues, and friends, I would also like to thank my family. During the COVID-19 pandemic in the last year of my PhD studies, I stayed at home and received a lot of care from them. I was able to really focus on finishing my thesis research and became much more productive than in the past 3 years. I am really grateful to their unconditional love. 

Finally, I would really like to thank my significant other Zhixin Lai. Since I met her in February 2020, she has been my soul mate. We shared everything in our life, worked together, and have learned a lot from each other. When we experienced obstacles and failures, we cheered each other up. When we saw opportunities and success, we expressed our happiness with each other and celebrated our achievements on the phone. She has helped me a lot in my career and gave me many helpful suggestions to improve the writing of my fellowship applications, job hunting, and PhD thesis. After almost a year and a half since we knew each other, the US-Canadian border was finally open. I finally had a chance to meet her in person. We sat together and prepared my thesis defense. Her company with me in my PhD thesis defense and my celebration dinner made it the most memorable day in my life. She has made me a more complete and mature person. I am indebted to her sincere love and really hope to experience more up and downs with her in the future.



%BNL: Craig Woody. Jin Huang. John Haggerty.

%LANL: ming

%Fermilab:

%SLAC: BJ


%CERN: Marrio Sitta, Ivan, Markus Keil, 



%I lived at Tang Hall and met many amazing people 


%Grad Lounge guys,

%PGSC Pizza social guys

%PPC bros

%MIT Heavy Ion group members



I am sorry that I may not be able to exhaust the list of everyone. But I really believe that I have learned a lot from all of you. I am sincerely grateful to your influences and friendships on my growth as a scientist and a person. 

%%%%%%%%%%%%%%%%%%%%%%%%%%%%%%%%%%%%%%%%%%%%%%%%%%%%%%%%%%%%%%%%%%%%%%
% -*-latex-*-
