%% This is an example first chapter.  You should put chapter/appendix that you
%% write into a separate file, and add a line \include{yourfilename} to
%% main.tex, where `yourfilename.tex' is the name of the chapter/appendix file.
%% You can process specific files by typing their names in at the 
%% \files=
%% prompt when you run the file main.tex through LaTeX.
\chapter{Reconstructed Objects}

The state-of-the-art CMS detector take a snapshot of each event and saves the detailed information of the collisions into datasets. In the datasets, we can access to event information with fully reconstructed objects including hits, tracks, muons, and vertex, which will be crucial for our data analysis to study B meson physics in heavy-ion collisions. Below, we will describe how these objects with physical meaning are reconstructed from electronic signal in the CMS detector.

\section{Event}

As mentioned previously, an event is defined as a snapshot of one collision at the LHC. Many particles are created in an event. Theoretically, to obtain the complete information of an event, we only need to know the position and momentum of each particle. Experimentally, we detect final state particles 

\section{Hit}

\section{Cluster}

\section{Track}

\section{Muon}

\section{Vertex}

\subsection{Primary Vertex}

\subsection{Secondary Vertex}

