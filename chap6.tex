%% This is an example first chapter.  You should put chapter/appendix that you
%% write into a separate file, and add a line \include{yourfilename} to
%% main.tex, where `yourfilename.tex' is the name of the chapter/appendix file.
%% You can process specific files by typing their names in at the 
%% \files=
%% prompt when you run the file main.tex through LaTeX.
\chapter{Conclusions}

With fully reconstructed $B^0_s$ and $B^+$ measurements in PbPb collisions, we can study the beauty hadronization mechanism and answer the questions raised in Section 2.6.

\section{$pp$ References and Theoretical Models}

Because our fully reconstructed B-meson analysis in $pp$, which serves as the reference for PbPb, is still ongoing, in order to understand our PbPb data, we need to add B-meson $pp$ measurements from other experiments at the LHC. The $pp$ references comparing with our PbPb measurement are described below:

\textbf{LHCb 7 TeV $pp$ results at $2 < |y| < 5$:} This reference is chosen because it is one of the most precise $B^0_s/B^+$ measurements with energy closest to 5.02 TeV in our analysis \cite{LHCbFF}. The original results are presented as the corrected yield ratio $\mathcal{R} = \frac{N(B^0_s \rightarrow J/\psi \phi)}{N(B^+ \rightarrow J/\psi K^+)} \cdot \frac{\epsilon(B^+ \rightarrow J/\psi K^+)}{\epsilon(B^0_s \rightarrow J/\psi \phi)}$ \cite{LHCbFF}. We multiply $\mathcal{R}$ by the branching ratios of $BR(B^0_s \rightarrow J/\psi \phi \rightarrow \mu^+\mu^- K^+ K^-)/BR(B^+ \rightarrow J/\psi K^+ \rightarrow \mu^+\mu^- K^+)$ and make them the same quantity as our $B^0_s/B^+$ measurement.

\textbf{ATLAS 7 TeV $pp$ results at $|y| < 2.5$:} This reference is chosen because it is measured over a rapidity range similar to our measurement range \cite{ATLASPPRef}. The original results are the ratios of the fragmentation fraction $f_s/f_d$. Using the isospin symmetry, we get $f_d = f_u$. So $f_s/f_d = f_s/f_u$. In addition, the ATLAS paper uses the QCD calculations $BF(QCD) = \frac{BR(B^0_s \rightarrow J/\psi \phi)}{BR(B^0 \rightarrow J/\psi K^{*0})} =$ 0.83 instead of directly quoting the PDG the branching ratios $BF(PDG) = \frac{BR(B^0_s \rightarrow J/\psi \phi)}{BR(B^0 \rightarrow J/\psi K^{*0})} =$ 0.85. Hence, we relate ATLAS $f_s/f_d$ to our $B^0_s/B^+$ ratio via $B^0_s/B^+ = BF(PDG)/BF(QCD) \times f_s/f_d$ and compare the ATLAS scaled $pp$ data to our data.


The $pp$ references of LHCb and ATLAS are at different rapidity. However, since the rapidity dependence is not significant in $B^0_s/B^+$ ratio as demonstrated in Figure \ref{BeautyFFLHCb} according to the LHCb publication \cite{LHCbFF}, assuming it is also insignificant in PbPb, we can use the $pp$ references at different rapidity ranges as references for our $B^0_s/B^+$ measurement.

In addition to the $pp$ references, we also include the theoretical predictions from TAMU (labeled as ``PbPb: TAMU'' in orange) and Cao, Sun, Ko (labeled as ``PbPb: Langevin'' in green) models which have been introduced in Section 1.6. Figure \ref{FinalResults} shows the comparison between our $B^0_s/B^+$ measurement with $pp$ references and theoretical model calculations. 

\begin{figure}[hbtp]
\begin{center}
\includegraphics[width=0.48\textwidth]{Figures/Chapter6/ratio_vsPt_ref1_1.pdf}
\includegraphics[width=0.48\textwidth]{Figures/Chapter6/ratio_vsCent_ref1.pdf}
\caption{The fully reconstructed $B^0_s/B^+$ ratio (right) as a function of $p_T$ (left) and PbPb event centrality (right) using the 2018 CMS dimuon PbPb dataset are shown above. Both plots include the ATLAS (magenta) and LHCb (blue) 7 TeV $pp$ references. The TAMU model (orange) has only $p_T$ dependent predictions shown on the left figure while the Cao, Sun, Ko model (green) has both $p_T$ and centrality predictions plotted on both figures.}
\label{FinalResults}
\end{center}
\end{figure}   
 

\section{Implications from the Experimental Data}

Figure \ref{FinalResults} conveys a lot of information. The physics messages from comparison of the PbPb data with the $p$p references and theoretical model prediction are discussed below:

%Lies above but with in about 1.5 sigma. not significant $p_T$ dependence. Agree reasonably well with both TAMU and Cao, Sun, Ko models. compatible with LHCb data

%List all the point here

\textbf{Substantial Uncertainties at Low $p_T$:} Both statistical and systematic uncertainties of $B^0_s/B^+$ ratio are large in $7 < p_T < 10$ GeV/c. They mostly come from $B^0_s$. However, we know that the statistics of $B^0_s$ in the $p_T$ range 7 - 10 GeV/c is indeed very small. In fact, from the FONLL calculation, we expect to get only about 12 $B^0_s$ signal candidates. Unfortunately, some of the systematic uncertainties, for instance, the one due to finite MC simulation statistics, which contributes a lot (26.5\%) to the total systematic uncertainties (46\%) can be in principle further reduced but reduced due to limited computing resources. 

\textbf{No Significant $p_T$ Dependence:} According to $B^0/B^+$ ratio as a function of B-meson $p_T$, apparently, there is no significant change of the central values for $p_T >$10 GeV/c. For 7 - 10 GeV/c, the central value jumps from 0.28 up to 0.51. However, the uncertainties of the measurement are also very large. Taking into account all the uncertainties, we do not observe significant $p_T$ dependence on the $B^0_s/B^+$ ratio.

\textbf{Good Agreement with Theoretical Models:} Both the TAMU and Cao, Sun Ko model calculations for the $B^0_s/B^+$ ratio vs $p_T$ agree reasonably well with the PbPb data. They both predict the trend of the central values of the data, which first decreases and then approaches flat values as $p_T$ increases. The TAMU Model always lies above the Cao, Sun, Ko Model due to the different applications of fragmentation hadronization mechanisms in the models.

However, we know that in the limit $p_T \rightarrow \infty$, the $B^0_s/B^+$ in PbPb collisions should be very similar to $pp$ collisions since the fast-moving beauty quarks traverse through the medium within a very short time and are not likely to combine with any quarks in the medium because they speeds are very different. Beauty quarks tend to radiate soft gluons and hadronize via fragmentation. Hence, fragmentation hadronization dominates in b-hadron production at very high $p_T$. 

As for the centrality measurement, the Cao, Sun, Ko model predictions are also reasonably consistent to our data in the range of 0 - 30\% and 0 - 90\%. However, in the peripheral 30 - 90\% collisions, the Cao, Sun, Ko model overshoots $B^0_s/B+$ ratio, roughly 2$\sigma$ above our data point that lies right on the $pp$ references.  


\textbf{Compatible to $pp$ References:} While the centers of the $B^0_s/B^+$ data points lie systematically above the $pp$ references, considering all uncertainties, they are within about 1$\sigma$ except the peripheral 30 - 90\% bin, which behaves like $pp$ because of the very small number of participants. However, it should note that the energy of $pp$ references are higher than the PbPb data. LHCb has reported that the $B^0_s/B^+$ ratio slightly increases as energy goes up \cite{LHCbFF}. Therefore, it would make the comparison much better if we could also perform $B^0_s/B^+$ measurement in $pp$ collisions with CMS and compare the results directly to the PbPb measurements.


%\textbf{Compatible to pp references:}

\section{Conclusions}


With the physics messages obtained from the discussions, we are prepared to answer the questions raised in Section 2.6 and draw the conclusions of our studies to this thesis:



\textbf{First Observation of $B^0_s$ in Nucleus-Nucleus Collisions:} In the analysis, we have fully reconstructed $B^0_s$ with greater 5$\sigma$ significances in all $p_T$ and centrality bins. Therefore, we have improved our measurements compared to the 2015 published results and declared the first observed fully reconstructed $B^0_s$ in heavy-ion collisions.

\textbf{Significant Improvement of the Previous Results:} We have successfully reproduced the published results using 2015 datasets with higher precision. Moreover, we measure $B^0_s/B^+$ as a function of centrality for the first time. In addition, thanks to the higher statistics of the dataset, we are able to measure four $p_T$ bins, providing more information about the $p_T$ dependence of the $B^0_s/B^+$ ratio. In our measurements, we find no significant $p_T$ dependence of $B^0_s/B^+$ down to at least 10 GeV/c. In addition, there is a hint of suppression of B-meson cross section in central collision compared to peripheral collisions, which will be confirmed with larger statistics later. 
 
\textbf{Inconclusive about Strangeness Enhancement:} There is a weak hint of potential strangeness enhancement for beauty quark hadronization in PbPb collisions, particularly at low $p_T$, compared to $pp$ collisions. The $B^0_s/B^+$ ratios in PbPb collisions are systematically higher than $pp$ collisions for about (1 -- 1.3)$\sigma$. However, the hint is not strong enough. We will need more statistics to confirm this hint in the future.

\textbf{The Fragmentation Hadronization Mechanism Alone Not Enough to Describe Our Data:} We can see that the quark coalescence effect must be considered because our data points lie systematically above the $pp$ references. Looking at the most central collision from 0 - 30 \%, the $B^0_s/B^+$ ratio is about 1.25$\sigma$ above the LHCb pp reference, which corresponds to about 80\% confidence. The explicit computation is shown as follows:

\begin{equation}
\% Dev = (0.3655 - 0.2353)/(0.3655 * \sqrt{0.23^2 + 0.172^2}) \simeq 1.25 
\end{equation}


\textbf{Not Enough Precision to Constrain Theoretical Models:} Base on the uncertainties of our data, we find that the theoretical models using quark coalescence to describe hadronization, for instance, the TAMU and Cao, Sun, Ko models, are all in reasonable agreement with the PbPb data, both in terms of central values and the decreasing trends as $p_T$ increases.

\textbf{Missing the B-meson Measurement in $pp$ with CMS as A Reference:} Currently, the B-meson analysis using the 2017 CMS $pp$ dataset is still working in progress. More results will be coming in the near future to answer the questions such as the beauty energy loss mechanism in the QGP and hadronization mechanism in small systems. Our B-meson $R_{AA}$ measurements will be able to constrain the heavy-quark spatial diffusion coefficient and the jet transport parameter to probe the inner workings of the QGP.

In conclusion, the larger PbPb datasets that should be accumulated in upcoming LHC Run 3 and high-luminosity (HL) LHC heavy-ion runs will provide greater precision and allow more differential B-meson measurements not only on traditional observables with but also on modern observables such as $B-\bar B$ angular correlations with more fully reconstructed b-hadron species such as $\Lambda_b$, $B^0_c$, and $\Omega_b$. In addition, the CMS MIP Timing Detector (MTD) upgrade \cite{CMSMTD} will allow us to perform hadronic PID. We will be able to fully reconstruct beauty hadrons down very low $p_T$ and carry out measurements with high precision. These future b-hadron measurements could help further investigate beauty hadronization in vacuum and QGP.


%The $B^0_s$ and $B^+$ mesons are studied with the CMS detector at the LHC via the reconstruction of the exclusive hadronic decay channels B\ and \Bplusdecayall.  The measurements are performed within the \PB\ mesons' fiducial region given by transverse momentum $\pt>10\GeVc$ for rapidity $\abs{y}<1.5$  and  $7<\pt<50\GeVc \,$ for $1.5<\abs{y}<2.4$. The first observation of the \PBzs\ meson in nucleus-nucleus collisions, with a statistical significance well surpassing five standard deviations, is attained. The production yields of \PBzs\ and \PBp\ mesons, scaled by the nuclear overlap function \TAA and the number of minimum bias events \NMB, in lead-lead (\PbPb) collisions at a center-of-mass energy of $5.02\TeV$ per nucleon pair are presented as functions of the meson \pt\ and for the first time of the event centrality. These results extend, and are compatible with, those previously reported by the CMS Collaboration~\cite{BsPbPbCMS,BpPbPbCMS}, and are based on a three-fold larger \PbPb\ data sample. The ratio of production yields of the two mesons in \PbPb\ collisions is determined and it is found to be statistically compatible with the corresponding ratio in proton-proton (\pp) collisions. The further investigation of possible hints of an enhancement of the ratio in \PbPb, relative to \pp, collisions will benefit from more precise \PbPb\ and \pp\ reference data taken at the same collision energy per nucleon. The larger \PbPb\ data sets that should be accumulated in upcoming high-luminosity LHC heavy ion runs will provide greater precision and could help to further characterize the mechanisms of beauty hadronization in heavy ion collisions.






%We have performed the 
%Conclude the message. Draw final based on the points. Answer question post previously

%More precise measurement. 



\section{Future Outlooks}

As mentioned previously, our B-meson data analysis in $pp$ collisions is still ongoing. Figure \ref{BPLow}, Figure \ref{BZLow}, and Figure \ref{BsLow} show our ongoing analysis of fully reconstructed $B^+$, $B^0$, and $B^0_s$ using the 2017 $pp$ datasets at $\sqrt {s_{NN}} = $ 5.02 TeV at very low $p_T$ respectfully


\begin{figure}[hbtp]
\begin{center}
\includegraphics[width=0.60\textwidth]{Figures/Chapter6/BPLow.pdf}
\caption{The fully reconstructed $B^+$ via the decay channel of $B^+\rightarrow J/\psi K^+ \rightarrow \mu^+\mu^- K^+$ in the $p_T$ range of 0 - 1 GeV/c using the full CMS 2017 $pp$ dataset is shown above. The statistical significance is about 6. The selection is optimized with the BDT algorithm using a subset of topological variables in PbPb $B^+$ studies.}
\label{BPLow}
\end{center}
\end{figure}   
 
 \begin{figure}[hbtp]
\begin{center}
\includegraphics[width=0.60\textwidth]{Figures/Chapter6/BZLow.pdf}
\caption{The fully reconstructed $B^0$ via the decay channel of $B^0\rightarrow J/\psi K^{0*} \rightarrow \mu^+\mu^- K \pi$ in the $p_T$ range of 2 - 4 GeV/c using the full CMS 2017 $pp$ dataset is shown above. The statistical significance is about 5.1. The selection is optimized with the BDT algorithm using a subset of topological variables in PbPb $B^0_s$ studies.}
\label{BZLow}
\end{center}
\end{figure}   

 \begin{figure}[hbtp]
\begin{center}
\includegraphics[width=0.60\textwidth]{Figures/Chapter6/BsLow.pdf}
\caption{The fully reconstructed $B^0_s$ via the decay channel of $B^0_s\rightarrow J/\psi \phi  \rightarrow \mu^+\mu^- K^+ K^-$ in the $p_T$ range of 2 - 4 GeV/c using the full CMS 2017 $pp$ dataset is shown above. The statistical significance is about 3.9. The selection is optimized with the BDT algorithm using a subset of topological variables in PbPb $B^0_s$ studies.}
\label{BsLow}
\end{center}
\end{figure}   

Thanks to the powerful machine learning algorithms, even without hadronic PID, very clear B-meson signals have still been observed down to $p_T =$ 0. The estimated significances are all greater than 4. With these significant signals, we can perform precise measurement on $B^+$ cross section in $pp$ collisions down to $p_T =$ 0, which allows us to study inclusive beauty production cross section. In addition, we will also be able to measure $B^0_s/B^+$ ratio down to 2 GeV/c. Finally, according to the multiplicity studies, we can also measure $B^0_s/B^+$ as a function of multiplicity up to about 150, which helps us answer many questions raised in Section 2.6. These fully B-meson measurements down to very low $p_T$ and up to very high multiplicity will be very important to study the beauty quark hadronization mechanisms in vacuum and QGP.

In the future era of LHC Run 3 and HL LHC, much more data will be collected to perform exciting measurements on fully reconstructed $B^+_c$ and $\Lambda_b$ hadrons. Meanwhile, at RHIC, as the sPHENIX experiment is taking data in 2023, we can also fully reconstruct b hadrons at lower energies to study a QGP medium at lower temperatures and higher baryon chemical potentials. The fully reconstructed b-hadron measurements at RHIC will be complementary to the measurements at the LHC. Together, these will help determine the heavy-quark diffusion coefficient at different temperatures, constrain the fundamental property of QGP $\eta/s$, and probe the inner workings of QGP. Lots of challenges and opportunities are waiting ahead for us to explore and overcome. A bright and exciting chapter of relativistic heavy-ion physics is just around the corner!



